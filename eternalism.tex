\documentclass{article}

\title{Anti-Eternalism and Computer Software}

\author{
Michael Roe \\
Bruce Christianson \\
}

\begin{document}

\maketitle

In this article, we will apply some concepts from Buddhist philosophy
(particulary anti-eternalism and ego fixation) to the problems
that are encountered when large teams of computer programmers
attempt to write very large computer programs, such as operating
systems. As readers of this paper may not be familiar with either
or both of these areas, we will attempt to explain matters in a way
that is accessible to a non-specialist reader. It will also lead
us to refer to a very diverse bibliography, from Machig Labdron (a
woman philosopher who lived in Tibet in the 11th century) to
Fred Brook's memoir of his time working as manager at IBM during
the development of the operating system OS360. At this point,
a reader who is familiar with Machig Labdron may wonder how we
are going to move the philosophical argument from human corpses
being devoured by vultures in Machig Labdron on to Fred Brook's
OS360,
and whether it might involve OS360 being devoured by vultures,
metaphorically if not literally.


****

According to Jaron Lanier, writing a small computer program of
a few dozen lines is a pleasant xperience, but working on really large
computer programs is horrible. Readers who are not computer programmers
may wonder is these computer programs are really as large as all
that, or whether we are engaging in rhetorical exaggeration.
The detailed numbers for Microsoft Windows (for example) may not be o
publically available, so here I will give the figures from a software
system that is snaller, but still large enough to potentially present
an unpleasant experience to anyone working on it. These numbers
are taken from the CHERI project, a devlopment of a microprocessor
chip with added security features, funded in part by the DARPA, 
the U.S. Defense Advanced Research Projects Agency.

***


According to Perdita Stevens, compilers are highly reliable pieces
of software, and if you a wondering whether the reason your program
doesn't work as you expect is an error in your own program or an
error in the compiler, the error is almost certainly in your own
program. This is excellent advice for beginning students who are
writing small computer programs as described by Jaron Lanier,
especially if the compiler and the CPU chip you are using is
of a type used by many other people. If there was an error in the
compiler, someone else would have complained about it already
and it would have been fixed\footnote{This is unduly optimistic}.
Stevens then goes on to add that this is not true if you also
making chnages to the compiler.

When working with experimental CPUs, the reason your program does
not work could be anywhere. It could be your program. It could be the i
compiler. It could
be in the chip design. It could be in the other compiler that
 the chip design into an electronic circuit.

Outline of the next bit: ``Government Furnished Equipment'' RISC-V
CPU design used in a friendly challenge in which DARPA invited
members of the public to try to break the security of a prototype
electronic voting terminal. (For the benefit of readers from 
countries like the UK where government elections are carried out
with paper ballot forms; some places in the United States use
a system where the voter pushes a button on a computer to vote
for one candidate or another. The concern here is that an error
somewhere in all those many many lines of computer code could
allow a suitable malicious person to trick the computer into
changing the result of who won the election).

(Description of bug)

At this point, we will return to Machig Labdron. According to
Machig, demons do not exist.

\cite{*}

\bibliographystyle{plain}
\bibliography{eternalism}

\end{document}
